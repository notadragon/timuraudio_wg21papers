%% Common header for WG21 proposals ? mainly taken from C++ standard draft source
%%

%%--------------------------------------------------
%% basics
\documentclass[a4paper,11pt,oneside,openany,final,article]{memoir}

\usepackage[american]
           {babel}        % needed for iso dates
\usepackage[iso,american]
           {isodate}      % use iso format for dates
\usepackage[final]
           {listings}     % code listings
\usepackage{longtable}    % auto-breaking tables
\usepackage{ltcaption}    % fix captions for long tables
\usepackage{relsize}      % provide relative font size changes
\usepackage{textcomp}     % provide \text{l,r}angle
\usepackage{underscore}   % remove special status of '_' in ordinary text
\usepackage{parskip}      % handle non-indented paragraphs "properly"
\usepackage{array}        % new column definitions for tables
\usepackage[normalem]{ulem}
\usepackage{enumitem}
\usepackage{color}        % define colors for strikeouts and underlines
\usepackage[dvipsnames]{xcolor}    % needed for blue links

\usepackage{amsmath}      % additional math symbols
\usepackage{mathrsfs}     % mathscr font
\usepackage[final]{microtype}
\usepackage{multicol}
\usepackage{lmodern}
\usepackage[T1]{fontenc}
\usepackage[pdftex, final]{graphicx}
\usepackage[pdftex,
            bookmarks=true,
            bookmarksnumbered=true,
            pdfpagelabels=true,
            pdfpagemode=UseOutlines,
            pdfstartview=FitH,
            linktocpage=true,
            colorlinks=true,
            plainpages=false,
            allcolors={blue}, 
            allbordercolors={white}
           ]{hyperref}
\usepackage{memhfixc}     % fix interactions between hyperref and memoir
\usepackage{url}  % urls in ref.bib
\usepackage{tabularx}  % don't use the C++ standard's fancy tables, they come with captions!

\pdfminorversion=5
\pdfcompresslevel=9
\pdfobjcompresslevel=2

\renewcommand\RSsmallest{5.5pt}  % smallest font size for relsize

%%--------------------------------------------------
%%--------------------------------------------------
%% Layout -- set overall page appearance

%%--------------------------------------------------
%%  set page size, type block size, type block position

\setlrmarginsandblock{2.245cm}{2.245cm}{*}
\setulmarginsandblock{2.5cm}{2.5cm}{*}

%%--------------------------------------------------
%%  set header and footer positions and sizes

\setheadfoot{\onelineskip}{2\onelineskip}
\setheaderspaces{*}{2\onelineskip}{*}

%%--------------------------------------------------
%%  make miscellaneous adjustments, then finish the layout
\setmarginnotes{7pt}{7pt}{0pt}
\checkandfixthelayout

%%--------------------------------------------------
%% If there is insufficient stretchable vertical space on a page,
%% TeX will not properly consider penalties for a good page break,
%% even if \raggedbottom (default) is in effect.
\addtolength{\topskip}{0pt plus 20pt}

%%--------------------------------------------------
%% Paragraph and bullet numbering

\newcounter{Paras}
\counterwithout{section}{chapter}
\setcounter{secnumdepth}{3}

\newcounter{Bullets1}[Paras]
\newcounter{Bullets2}[Bullets1]
\newcounter{Bullets3}[Bullets2]
\newcounter{Bullets4}[Bullets3]

\makeatletter
\newcommand{\parabullnum}[2]{%
\stepcounter{#1}%
\noindent\makebox[0pt][l]{\makebox[#2][r]{%
\scriptsize\raisebox{.7ex}%
{%
\ifnum \value{Paras}>0
\ifnum \value{Bullets1}>0 (\fi%
                          \arabic{Paras}%
\ifnum \value{Bullets1}>0 .\arabic{Bullets1}%
\ifnum \value{Bullets2}>0 .\arabic{Bullets2}%
\ifnum \value{Bullets3}>0 .\arabic{Bullets3}%
\fi\fi\fi%
\ifnum \value{Bullets1}>0 )\fi%
\fi%
}%
\hspace{\@totalleftmargin}\quad%
}}}
\makeatother

\def\pnum{\parabullnum{Paras}{0pt}}

%%--------------------------------------------------
%%--------------------------------------------------
%% Styles
%!TEX root = std.tex
%% styles.tex -- set styles for:
%     chapters
%     pages
%     footnotes

%%--------------------------------------------------
%%  create chapter style

\makechapterstyle{cppstd}{%
  \renewcommand{\beforechapskip}{\onelineskip}
  \renewcommand{\afterchapskip}{\onelineskip}
  \renewcommand{\chapternamenum}{}
  \renewcommand{\chapnamefont}{\chaptitlefont}
  \renewcommand{\chapnumfont}{\chaptitlefont}
  \renewcommand{\printchapternum}{\chapnumfont\thechapter\quad}
  \renewcommand{\afterchapternum}{}
}

%%--------------------------------------------------
%%  create page styles




%%--------------------------------------------------
% set style for main text
\setlength{\parindent}{0pt}
\setlength{\parskip}{1ex}

%%--------------------------------------------------
%% change list item markers to number and em-dash

\renewcommand{\labelitemi}{---\parabullnum{Bullets1}{\labelsep}}
\renewcommand{\labelitemii}{---\parabullnum{Bullets2}{\labelsep}}
\renewcommand{\labelitemiii}{---\parabullnum{Bullets3}{\labelsep}}
\renewcommand{\labelitemiv}{---\parabullnum{Bullets4}{\labelsep}}



%%--------------------------------------------------
%% override some functions from the listings package to avoid bad page breaks
%% (copied verbatim from listings.sty version 1.6 except where commented)
\makeatletter

\def\lst@Init#1{%
    \begingroup
    \ifx\lst@float\relax\else
        \edef\@tempa{\noexpand\lst@beginfloat{lstlisting}[\lst@float]}%
        \expandafter\@tempa
    \fi
    \ifx\lst@multicols\@empty\else
        \edef\lst@next{\noexpand\multicols{\lst@multicols}}
        \expandafter\lst@next
    \fi
    \ifhmode\ifinner \lst@boxtrue \fi\fi
    \lst@ifbox
        \lsthk@BoxUnsafe
        \hbox to\z@\bgroup
             $\if t\lst@boxpos \vtop
        \else \if b\lst@boxpos \vbox
        \else \vcenter \fi\fi
        \bgroup \par\noindent
    \else
        \lst@ifdisplaystyle
            \lst@EveryDisplay
            % make penalty configurable
            \par\lst@beginpenalty
            \vspace\lst@aboveskip
        \fi
    \fi
    \normalbaselines
    \abovecaptionskip\lst@abovecaption\relax
    \belowcaptionskip\lst@belowcaption\relax
    \lst@MakeCaption t%
    \lsthk@PreInit \lsthk@Init
    \lst@ifdisplaystyle
        \global\let\lst@ltxlabel\@empty
        \if@inlabel
            \lst@ifresetmargins
                \leavevmode
            \else
                \xdef\lst@ltxlabel{\the\everypar}%
                \lst@AddTo\lst@ltxlabel{%
                    \global\let\lst@ltxlabel\@empty
                    \everypar{\lsthk@EveryLine\lsthk@EveryPar}}%
            \fi
        \fi
        % A section heading might have set \everypar to apply a \clubpenalty
        % to the following paragraph, changing \everypar in the process.
        % Unconditionally overriding \everypar is a bad idea.
        % \everypar\expandafter{\lst@ltxlabel
        %                      \lsthk@EveryLine\lsthk@EveryPar}%
    \else
        \everypar{}\let\lst@NewLine\@empty
    \fi
    \lsthk@InitVars \lsthk@InitVarsBOL
    \lst@Let{13}\lst@MProcessListing
    \let\lst@Backslash#1%
    \lst@EnterMode{\lst@Pmode}{\lst@SelectCharTable}%
    \lst@InitFinalize}

\def\lst@DeInit{%
    \lst@XPrintToken \lst@EOLUpdate
    \global\advance\lst@newlines\m@ne
    \lst@ifshowlines
        \lst@DoNewLines
    \else
        \setbox\@tempboxa\vbox{\lst@DoNewLines}%
    \fi
    \lst@ifdisplaystyle \par\removelastskip \fi
    \lsthk@ExitVars\everypar{}\lsthk@DeInit\normalbaselines\normalcolor
    \lst@MakeCaption b%
    \lst@ifbox
        \egroup $\hss \egroup
        \vrule\@width\lst@maxwidth\@height\z@\@depth\z@
    \else
        \lst@ifdisplaystyle
            % make penalty configurable
            \par\lst@endpenalty
            \vspace\lst@belowskip
        \fi
    \fi
    \ifx\lst@multicols\@empty\else
        \def\lst@next{\global\let\@checkend\@gobble
                      \endmulticols
                      \global\let\@checkend\lst@@checkend}
        \expandafter\lst@next
    \fi
    \ifx\lst@float\relax\else
        \expandafter\lst@endfloat
    \fi
    \endgroup}


\def\lst@NewLine{%
    \ifx\lst@OutputBox\@gobble\else
        \par
        % add configurable penalties
        \lst@ifeolsemicolon
          \lst@semicolonpenalty
          \lst@eolsemicolonfalse
        \else
          \lst@domidpenalty
        \fi
        % Manually apply EveryLine and EveryPar; do not depend on \everypar
        \noindent \hbox{}\lsthk@EveryLine%
        % \lsthk@EveryPar uses \refstepcounter which balloons the PDF
    \fi
    \global\advance\lst@newlines\m@ne
    \lst@newlinetrue}

% new macro for empty lines, avoiding an \hbox that cannot be discarded
\def\lst@DoEmptyLine{%
  \ifvmode\else\par\fi\lst@emptylinepenalty
  \vskip\parskip
  \vskip\baselineskip
  % \lsthk@EveryLine has \lst@parshape, i.e. \parshape, which causes an \hbox
  % \lsthk@EveryPar increments line counters; \refstepcounter balloons the PDF
  \global\advance\lst@newlines\m@ne
  \lst@newlinetrue}

\def\lst@DoNewLines{
    \@whilenum\lst@newlines>\lst@maxempty \do
        {\lst@ifpreservenumber
            \lsthk@OnEmptyLine
            \global\advance\c@lstnumber\lst@advancelstnum
         \fi
         \global\advance\lst@newlines\m@ne}%
    \@whilenum \lst@newlines>\@ne \do
        % special-case empty printing of lines
        {\lsthk@OnEmptyLine\lst@DoEmptyLine}%
    \ifnum\lst@newlines>\z@ \lst@NewLine \fi}

% add keys for configuring before/end vertical penalties
\lst@Key{beginpenalty}\relax{\def\lst@beginpenalty{\penalty #1}}
\let\lst@beginpenalty\@empty
\lst@Key{midpenalty}\relax{\def\lst@midpenalty{\penalty #1}}
\let\lst@midpenalty\@empty
\lst@Key{endpenalty}\relax{\def\lst@endpenalty{\penalty #1}}
\let\lst@endpenalty\@empty
\lst@Key{emptylinepenalty}\relax{\def\lst@emptylinepenalty{\penalty #1}}
\let\lst@emptylinepenalty\@empty
\lst@Key{semicolonpenalty}\relax{\def\lst@semicolonpenalty{\penalty #1}}
\let\lst@semicolonpenalty\@empty

\lst@AddToHook{InitVars}{\let\lst@domidpenalty\@empty}
\lst@AddToHook{InitVarsEOL}{\let\lst@domidpenalty\lst@midpenalty}

% handle semicolons and closing braces (could be in \lstdefinelanguage as well)
\def\lst@eolsemicolontrue{\global\let\lst@ifeolsemicolon\iftrue}
\def\lst@eolsemicolonfalse{\global\let\lst@ifeolsemicolon\iffalse}
\lst@AddToHook{InitVars}{
  \global\let\lst@eolsemicolonpending\@empty
  \lst@eolsemicolonfalse
}
% If we found a semicolon or closing brace while parsing the current line,
% inform the subsequent \lst@NewLine about it for penalties.
\lst@AddToHook{InitVarsEOL}{%
  \ifx\lst@eolsemicolonpending\relax
    \lst@eolsemicolontrue
    \global\let\lst@eolsemicolonpending\@empty
  \fi%
}
\lst@AddToHook{SelectCharTable}{%
  % In theory, we should only detect trailing semicolons or braces,
  % but that would require un-doing the marking for any other character.
  % The next best thing is to undo the marking for closing parentheses,
  % because loops or if statements are the only places where we will
  % reasonably have a semicolon in the middle of a line, and those all
  % end with a closing parenthesis.
  \lst@DefSaveDef{41}\lstsaved@closeparen{%    handle closing parenthesis
    \lstsaved@closeparen
    \ifnum\lst@mode=\lst@Pmode    % regular processing mode (not a comment)
      \global\let\lst@eolsemicolonpending\@empty  % undo semicolon setting
    \fi%
  }%
  \lst@DefSaveDef{59}\lstsaved@semicolon{%     handle semicolon
    \lstsaved@semicolon
    \ifnum\lst@mode=\lst@Pmode    % regular processing mode (not a comment)
      \global\let\lst@eolsemicolonpending\relax
    \fi%
  }%
  \lst@DefSaveDef{125}\lstsaved@closebrace{%   handle closing brace
    \lst@eolsemicolonfalse        % do not break before a closing brace
    \lstsaved@closebrace          % might invoke \lst@NewLine
    \ifnum\lst@mode=\lst@Pmode    % regular processing mode (not a comment)
      \global\let\lst@eolsemicolonpending\relax
    \fi%
  }%
}

\makeatother


%%--------------------------------------------------
%%--------------------------------------------------
%% Macros
%!TEX root = std.tex
% Definitions and redefinitions of special commands

%%--------------------------------------------------
%% Difference markups
\definecolor{addclr}{rgb}{0,0.5,0.1}
\definecolor{remclr}{rgb}{1,0,0}
\definecolor{noteclr}{rgb}{0,0,1}

\renewcommand{\added}[1]{\textcolor{addclr}{\uline{#1}}}
\newcommand{\removed}[1]{\textcolor{remclr}{\sout{#1}}}
\renewcommand{\changed}[2]{\removed{#1}\added{#2}}

\newcommand{\nbc}[1]{[#1]\ }
\newcommand{\addednb}[2]{\added{\nbc{#1}#2}}
\newcommand{\removednb}[2]{\removed{\nbc{#1}#2}}
\newcommand{\changednb}[3]{\removednb{#1}{#2}\added{#3}}
\newcommand{\remitem}[1]{\item\removed{#1}}

\newcommand{\ednote}[1]{\textcolor{noteclr}{[Editor's note: #1] }}
% \newcommand{\ednote}[1]{}

\newenvironment{addedblock}
{
\color{addclr}
}
{
\color{black}
}
\newenvironment{removedblock}
{
\color{remclr}
}
{
\color{black}
}

%%--------------------------------------------------
% General code style
\newcommand{\CodeStyle}{\ttfamily}
\newcommand{\CodeStylex}[1]{\texttt{#1}}

% Code and definitions embedded in text.
\newcommand{\tcode}[1]{\CodeStylex{#1}}
\newcommand{\techterm}[1]{\textit{#1}}
\newcommand{\defnx}[2]{\indexdefn{#2}\textit{#1}}
\newcommand{\defn}[1]{\defnx{#1}{#1}}
\newcommand{\term}[1]{\textit{#1}}
\newcommand{\grammarterm}[1]{\textit{#1}}
\newcommand{\grammartermnc}[1]{\textit{#1}\nocorr}
\newcommand{\placeholder}[1]{\textit{#1}}
\newcommand{\placeholdernc}[1]{\textit{#1\nocorr}}

%%--------------------------------------------------
%% allow line break if needed for justification
\newcommand{\brk}{\discretionary{}{}{}}

%%--------------------------------------------------
%% Macros for funky text
\newcommand{\Cpp}{\texorpdfstring{C\kern-0.05em\protect\raisebox{.35ex}{\textsmaller[2]{+\kern-0.05em+}}}{C++}}
\newcommand{\CppIII}{\Cpp{} 2003}
\newcommand{\CppXI}{\Cpp{} 2011}
\newcommand{\CppXIV}{\Cpp{} 2014}
\newcommand{\CppXVII}{\Cpp{} 2017}
\newcommand{\opt}[1]{\ifthenelse{\equal{#1}{}}
    {\PackageError{main}{argument must not be empty}{}}
    {#1\ensuremath{_\mathit{opt}}}}
\newcommand{\dcr}{-{-}}
\newcommand{\bigoh}[1]{\ensuremath{\mathscr{O}(#1)}}

% Make all tildes a little larger to avoid visual similarity with hyphens.
\renewcommand{\~}{\textasciitilde}
\let\OldTextAsciiTilde\textasciitilde
\renewcommand{\textasciitilde}{\protect\raisebox{-0.17ex}{\larger\OldTextAsciiTilde}}
\newcommand{\caret}{\char`\^}

%%--------------------------------------------------
%% States and operators
\newcommand{\state}[2]{\tcode{#1}\ensuremath{_{#2}}}
\newcommand{\bitand}{\ensuremath{\, \mathsf{bitand} \,}}
\newcommand{\bitor}{\ensuremath{\, \mathsf{bitor} \,}}
\newcommand{\xor}{\ensuremath{\, \mathsf{xor} \,}}
\newcommand{\rightshift}{\ensuremath{\, \mathsf{rshift} \,}}
\newcommand{\leftshift}[1]{\ensuremath{\, \mathsf{lshift}_#1 \,}}

%% Notes and examples
\newcommand{\noteintro}[1]{[\,\textit{#1:}\space}
\newcommand{\noteoutro}[1]{\textit{\,---\,end #1}\,]}
\newenvironment{note}[1][Note]{\noteintro{#1}}{\noteoutro{note}\space}
\newenvironment{example}[1][Example]{\noteintro{#1}}{\noteoutro{example}\space}

%% Library function descriptions
\newcommand{\Fundescx}[1]{\textit{#1}}
\newcommand{\Fundesc}[1]{\Fundescx{#1:}\space}
\newcommand{\required}{\Fundesc{Required behavior}}
\newcommand{\requires}{\Fundesc{Requires}}
\newcommand{\effects}{\Fundesc{Effects}}
\newcommand{\postconditions}{\Fundesc{Postconditions}}
\newcommand{\returns}{\Fundesc{Returns}}
\newcommand{\throws}{\Fundesc{Throws}}
\newcommand{\default}{\Fundesc{Default behavior}}
\newcommand{\complexity}{\Fundesc{Complexity}}
\newcommand{\remarks}{\Fundesc{Remarks}}
\newcommand{\errors}{\Fundesc{Error conditions}}
\newcommand{\sync}{\Fundesc{Synchronization}}
\newcommand{\implimits}{\Fundesc{Implementation limits}}
\newcommand{\replaceable}{\Fundesc{Replaceable}}
\newcommand{\returntype}{\Fundesc{Return type}}
\newcommand{\cvalue}{\Fundesc{Value}}
\newcommand{\ctype}{\Fundesc{Type}}
\newcommand{\ctypes}{\Fundesc{Types}}
\newcommand{\dtype}{\Fundesc{Default type}}
\newcommand{\ctemplate}{\Fundesc{Class template}}
\newcommand{\templalias}{\Fundesc{Alias template}}

%% Cross reference
\newcommand{\xref}{\textsc{See also:}\space}

%% Inline parenthesized reference
\newcommand{\iref}[1]{\nolinebreak[3] (\ref{#1})}

%% NTBS, etc.
\newcommand{\NTS}[1]{\textsc{#1}}
\newcommand{\ntbs}{\NTS{ntbs}}
\newcommand{\ntmbs}{\NTS{ntmbs}}
% The following are currently unused:
% \newcommand{\ntwcs}{\NTS{ntwcs}}
% \newcommand{\ntcxvis}{\NTS{ntc16s}}
% \newcommand{\ntcxxxiis}{\NTS{ntc32s}}

%% Code annotations
\newcommand{\EXPO}[1]{\textit{#1}}
\newcommand{\expos}{\EXPO{exposition only}}
\newcommand{\impdef}{\EXPO{implementation-defined}}
\newcommand{\impdefnc}{\EXPO{implementation-defined\nocorr}}
\newcommand{\impdefx}[1]{\indeximpldef{#1}\EXPO{implementation-defined}}
\newcommand{\notdef}{\EXPO{not defined}}

\newcommand{\UNSP}[1]{\textit{\texttt{#1}}}
\newcommand{\UNSPnc}[1]{\textit{\texttt{#1}\nocorr}}
\newcommand{\unspec}{\UNSP{unspecified}}
\newcommand{\unspecnc}{\UNSPnc{unspecified}}
\newcommand{\unspecbool}{\UNSP{unspecified-bool-type}}
\newcommand{\seebelow}{\UNSP{see below}}
\newcommand{\seebelownc}{\UNSPnc{see below}}
\newcommand{\unspecuniqtype}{\UNSP{unspecified unique type}}
\newcommand{\unspecalloctype}{\UNSP{unspecified allocator type}}

\newcommand{\EXPLICIT}{\textit{\texttt{EXPLICIT}\nocorr}}

%% Manual insertion of italic corrections, for aligning in the presence
%% of the above annotations.
\newlength{\itcorrwidth}
\newlength{\itletterwidth}
\newcommand{\itcorr}[1][]{%
 \settowidth{\itcorrwidth}{\textit{x\/}}%
 \settowidth{\itletterwidth}{\textit{x\nocorr}}%
 \addtolength{\itcorrwidth}{-1\itletterwidth}%
 \makebox[#1\itcorrwidth]{}%
}

%% Double underscore
\newcommand{\ungap}{\kern.5pt}
\newcommand{\unun}{\textunderscore\ungap\textunderscore}
\newcommand{\xname}[1]{\tcode{\unun\ungap#1}}
\newcommand{\mname}[1]{\tcode{\unun\ungap#1\ungap\unun}}

%% An elided code fragment, /* ... */, that is formatted as code.
%% (By default, listings typeset comments as body text.)
%% Produces 9 output characters.
\newcommand{\commentellip}{\tcode{/* ...\ */}}

%% Ranges
\newcommand{\Range}[4]{\tcode{#1#3,\penalty2000{} #4#2}}
\newcommand{\crange}[2]{\Range{[}{]}{#1}{#2}}
\newcommand{\brange}[2]{\Range{(}{]}{#1}{#2}}
\newcommand{\orange}[2]{\Range{(}{)}{#1}{#2}}
\newcommand{\range}[2]{\Range{[}{)}{#1}{#2}}

%% Change descriptions
\newcommand{\diffdef}[1]{\hfill\break\textbf{#1:}\space}
\newcommand{\diffref}[1]{\pnum\textbf{Affected subclause:} \ref{#1}}
\newcommand{\change}{\diffdef{Change}}
\newcommand{\rationale}{\diffdef{Rationale}}
\newcommand{\effect}{\diffdef{Effect on original feature}}
\newcommand{\difficulty}{\diffdef{Difficulty of converting}}
\newcommand{\howwide}{\diffdef{How widely used}}

%% Miscellaneous
\newcommand{\uniquens}{\placeholdernc{unique}}
\newcommand{\stage}[1]{\item[Stage #1:]}
\newcommand{\doccite}[1]{\textit{#1}}
\newcommand{\cvqual}[1]{\textit{#1}}
\newcommand{\cv}{\cvqual{cv}}
\newcommand{\numconst}[1]{\textsl{#1}}
\newcommand{\logop}[1]{{\footnotesize #1}}

%%--------------------------------------------------
%% Environments for code listings.

% We use the 'listings' package, with some small customizations.  The
% most interesting customization: all TeX commands are available
% within comments.  Comments are set in italics, keywords and strings
% don't get special treatment.

\lstset{language=C++,
        basicstyle=\small\CodeStyle,
        keywordstyle=,
        stringstyle=,
        xleftmargin=1em,
        showstringspaces=false,
        commentstyle=\itshape\rmfamily,
        columns=fullflexible,
        keepspaces=true,
        texcl=true}

% Our usual abbreviation for 'listings'.  Comments are in
% italics.  Arbitrary TeX commands can be used if they're
% surrounded by @ signs.
\newcommand{\CodeBlockSetup}{
 \lstset{escapechar=@, aboveskip=\parskip, belowskip=0pt,
         midpenalty=500, endpenalty=-50,
         emptylinepenalty=-250, semicolonpenalty=0}
 \renewcommand{\tcode}[1]{\textup{\CodeStylex{##1}}}
 \renewcommand{\techterm}[1]{\textit{\CodeStylex{##1}}}
 \renewcommand{\term}[1]{\textit{##1}}
 \renewcommand{\grammarterm}[1]{\textit{##1}}
}

\lstnewenvironment{codeblock}{\CodeBlockSetup}{}

% An environment for command / program output that is not C++ code.
\lstnewenvironment{outputblock}{\lstset{language=}}{}

% A code block in which single-quotes are digit separators
% rather than character literals.
\lstnewenvironment{codeblockdigitsep}{
 \CodeBlockSetup
 \lstset{deletestring=[b]{'}}
}{}

% Permit use of '@' inside codeblock blocks (don't ask)
\makeatletter
\newcommand{\atsign}{@}
\makeatother

%%--------------------------------------------------
%% Indented text
\newenvironment{indented}[1][]
{\begin{indenthelper}[#1]\item\relax}
{\end{indenthelper}}

%%--------------------------------------------------
%% Library item descriptions
\lstnewenvironment{itemdecl}
{
 \lstset{escapechar=@,
 xleftmargin=0em,
 midpenalty=500,
 semicolonpenalty=-50,
 endpenalty=3000,
 aboveskip=2ex,
 belowskip=0ex	% leave this alone: it keeps these things out of the
				% footnote area
 }
}
{
}

\newenvironment{itemdescr}
{
 \begin{indented}[beginpenalty=3000, endpenalty=-300]}
{
 \end{indented}
}

%%--------------------------------------------------
%% add special hyphenation rules
\hyphenation{tem-plate ex-am-ple in-put-it-er-a-tor name-space name-spaces non-zero}

%%--------------------------------------------------
%% turn off all ligatures inside \texttt
\DisableLigatures{encoding = T1, family = tt*}






% Footnotes at bottom of page:
 \usepackage[bottom]{footmisc} 

% Table going across a page: 
 \usepackage{longtable}

 % Start sections at 0
% \setcounter{section}{-1}

% color boxes
\usepackage{tikz,lipsum,lmodern}
\usepackage[most]{tcolorbox}

%%%%%%%%%%%%%%%%%%%%%%%%%%%%%%%%%%%%%%%%%%%%%%%%

%TABLE OF CONTENTS SETTINGS

\usepackage{titlesec}
\usepackage{tocloft}

% Custom ToC layout because the default sucks
\cftsetindents{section}{0in}{0.24in}
\cftsetindents{subsection}{0.24in}{0.34in}
\cftsetindents{subsubsection}{0.58in}{0.44in}

% Needed later to reduce the ToC depth mid document
\newcommand{\changelocaltocdepth}[1]{%
  \addtocontents{toc}{\protect\setcounter{tocdepth}{#1}}%
  \setcounter{tocdepth}{#1}%
}

\setcounter{tocdepth}{3}

%%%%%%%%%%%%%%%%%%%%%%%%%%%%%%%%%%%%%%%%%%%%%%%%

\begin{document}
\title{Postconditions odr-using a parameter \\ modified in an overriding function}
\author{
Timur Doumler \small(\href{mailto:papers@timur.audio}{papers@timur.audio}) \\
Joshua Berne \small(\href{mailto:jberne4@bloomberg.net}{jberne4@bloomberg.net}) \\
}
\date{}
\maketitle

\begin{tabular}{ll}
Document \#: & D3484R2 \\
Date: &2024-11-09 \\
Project: & Programming Language C++ \\
Audience: & SG21, EWG
\end{tabular}

\begin{abstract}
This paper considers the case where an overridden function odr-uses a non-reference function parameter in its postcondition assertion, and then an overriding function drops \tcode{const} on the declaration of that parameter, rendering the postcondition assertion in the overridden function significantly less useful. We propose several alternatives for how to address this problem in the Contracts MVP \cite{P2900R10}.
\end{abstract}

%%%%%%%%%%%%%%%%%%%%%%%%%%%%%%%%%%%%%%%%%%%%%

%\tableofcontents*
%\pagebreak

%%%%%%%%%%%%%%%%%%%%%%%%%%%%%%%%%%%%%%%%%%%%%

This paper is the first part of a trilogy of papers dealing with known issues in the Contracts MVP \cite{P2900R10} regarding postconditions odr-using non-reference function parameters:
\begin{itemize}
\item \cite{D3484R2} Postconditions odr-using a parameter modified in an overriding function;
\item \cite{P3487R0} Postconditions odr-using a parameter that may be passed in registers;
\item \cite{P3489R0} Postconditions odr-using a parameter of dependent type.
\end{itemize}
These issues should be considered together, and ideally resolved in a consistent way.

\section{Background}
\label{bg}

\cite{P2900R10} Section 3.4.4 specifies that if a non-reference function parameter is odr-used in a postcondition assertion, that function parameter must be \tcode{const} on \emph{all} declarations of that function, otherwise the program is ill-formed.  Importantly, this requirement on all declarations of the function includes its \emph{defining} declaration; thus, the implementation of the function can be assumed to not modify the value of that parameter.

The rationale for this rule is that allowing the implementation to modify a parameter that is later used to check the postcondition would render that check significantly less useful as it would be impossible to reason about the postcondition without taking into account what happens in the function definition. Consider the following function declaration:
\begin{codeblock}
// Returns: a number guaranteed to be greater or equal to the number passed in
int f(const int i) post (r: r >= i);
\end{codeblock}
Now, if we could drop the \tcode{const} on the defining declaration of \tcode{f}, we could write a dummy implementation of \tcode{f} as follows:
\begin{codeblock}
int f(int i) {  // \tcode{i} is not \tcode{const}
  i = 0;
  return i;
}
\end{codeblock}
Such an implementation blatantly fails to satisfy the postcondition of \tcode{f}, and is therefore incorrect. However, this defect cannot be caught by checking the postcondition assertion of \tcode{f}, because its predicate is now \tcode{0 >= 0} which always evaluates to \tcode{true}:
\begin{codeblock}
void test() {
  int j = f(3); // no contract violation detected
  // precondition does not hold --- \tcode{j} is now \tcode{0}, which is smaller than \tcode{3}!
}
\end{codeblock}
In the above scenario, the implementation of \tcode{f} explicitly modifies the parameter. However, an even more dangerous variant of this scenario is when such modifications happen \emph{implicitly}. In particular, if the parameter is not \tcode{const} on the defining declaration of \tcode{f}, it may be returned by value, which will perform an implicit move, i.e. the postcondition assertion would observe the parameter object in a moved-from state.

For these reasons, \cite{P2900R10} makes the above implementation of \tcode{f} ill-formed: \tcode{const} cannot be dropped from any function parameter odr-used in a postcondition assertion.

Further, \cite{P2900R10} Section 3.4.4 specifies that if a non-reference function parameter is odr-used in a postcondition assertion, and that function is implemented as a coroutine, the program is ill-formed, \emph{even if} all such function parameters have been declared \tcode{const} by the user.

The rationale for this rule is that a coroutine will move-from its function parameters to initialise the parameter copies in the coroutine frame, and therefore the function parameters of a coroutine are effectively not \tcode{const}, even if declared as such by the user (see \cite{P3387R0} for discussion). This case is thus notionally similar to the previous case where \tcode{const} was dropped from the parameter declaration in the implementation.

Together, these rules provide a static guarantee that, if a parameter is odr-used in a postcondition assertion, this parameter object \emph{will not be modified} between the call to a function and the evaluation of its postcondition assertions. An additional benefit of this design is that a static analysis tool can reason about the value of a function parameter in a postcondition without having to  take into account the function definition, which will often be either too complicated to analyse or unavailable at the call site.

Reference parameters are excluded from the above restrictions because references refer to objects declared elsewhere, and the value of those objects when the function call completes are still relevant and available to the caller because those objects can still have outside references known to the caller.  
There is no expectation that the value will remain unchanged after the function body has executed, and many functions that pass an object through a modifiable reference do so with the exact intention of modifying that object; therefore, such parameters would not be used in a postcondition assertion with that expectation in mind.

\section{The problem}

Let us now slightly modify the example above by making \tcode{f} a virtual function:
\begin{codeblock}
struct Base {
  virtual int f(const int i) post (r: r >= i);
};
\end{codeblock}
Note that if we override a virtual function, C++ allows dropping \tcode{const} from the parameter declaration in the overriding function, and \cite{P2900R10} currently does not have any provision to make such an override ill-formed:
\begin{codeblock}
struct Derived : Base {
  int f(int i) override; // OK
};
\end{codeblock}
This means that we can implement \tcode{Derived::f} such that it modifies the value of the parameter:
\begin{codeblock}
int Derived::f(int i) {
  i = 0;
  return i;
};
\end{codeblock}
\cite{P2900R10} Section 3.5.3 specifies the semantics of precondition and postcondition assertions on virtual functions: in a virtual function call, the function contract assertions of both the statically called function \tcode{Base::f} and the final overrider \tcode{Derived::f} are checked (see \cite{P3097R0} for discussion). However, if we now call \tcode{Derived::f} through a reference to \tcode{Base}, we are in for an unpleasant surprise:
\begin{codeblock}
void test(Base& b) {
  int j = b.f(3);  // no contract violation detected
  // precondition does not hold --- \tcode{j} is now \tcode{0}, which is smaller than \tcode{3}!
}

int main() {
  Derived d;
  test(d);
}
\end{codeblock}

In the program above, even if the postcondition assertion of \tcode{Base::f} is checked, the fact that the implementation of \tcode{Derived::f} does \emph{not} satisfy the postcondition of \tcode{Base::f} is not caught, because the parameter \tcode{i} has been modified in \tcode{Derived::f}, rendering the postcondition assertion of \tcode{Base::f} meaningless. This is exactly the scenario that the specification in \cite{P2900R10} Section 3.4.4 seeks to avoid; nevertheless, the code above is well-formed according to the current specification.

The more dangerous variant of this scenario that a parameter might be modified \emph{implicitly}, for example moved-from when returned by value, is also possible and well-formed:
\begin{codeblock}
struct Base2 {
  virtual std::string f(const std::string p) post(r : r.starts_with(p));
};

class Derived2 : public Base2 {
  std::string f(std::string p) override {
    return p;
  }
};
\end{codeblock}
In the above example, returning \tcode{p} from \tcode{Derived2::f} results in an implicit move from \tcode{p}, leading to the postcondition being checked with \tcode{p} being in a moved-from state and thus spuriously failing.

Due to examples such as this, the requirement to add \tcode{const} was added to the design that evolved into \cite{P2900R10} long ago by SG21. For this same reason even when a coding style dictates no modification of function parameters there is still a significant risk of bugs unless \tcode{const} is actually required on the definitions.

Unlike other cases that may lead to a \tcode{const} parameter being modified in the function body, namely when \tcode{const} is dropped on a subsequent declaration of the \emph{same} function, or when the function is implemented as a coroutine (which are both ill-formed in \cite{P2900R10}), in this case \tcode{Base::f} does not know whether there are any functions overriding it, or how the corresponding parameter in those overriding functions is declared.

\section{Possible solutions}

We are aware of six possible approaches to dealing with this problem. These are, from most to least restrictive:

% custom enumerators with V prefix:
\renewcommand\labelenumi{V\arabic{enumi}.}
\renewcommand\theenumi\labelenumi
\begin{enumerate}
\item Disallow odr-using any non-reference function parameter in a postcondition assertion that applies to a virtual function, regardless of whether that parameter is declared \tcode{const}, unless that function is marked \tcode{final} or is a member function of a class marked \tcode{final}.
\item Require that if a non-reference parameter is odr-used in a postcondition assertion on a virtual function, that parameter must also be declared \tcode{const} in every declaration of every overriding function.
\item Require that if a non-reference parameter is odr-used in a postcondition assertion on a virtual function, that parameter must also be declared \tcode{const} in the definition of every overriding function. The \tcode{const} may still be dropped in any non-defining declarations of any overriding functions.
\item Require that if a non-reference parameter is odr-used in a postcondition assertion on \emph{any} function, that parameter be declared \tcode{const} in the definition of that function and any overriding functions. The \tcode{const} may be dropped in any non-defining declarations of that function (which is a relaxation of the current rule in \cite{P2900R10}) and any overriding functions.
\item Allow overriding functions to drop \tcode{const} from a non-reference function parameter, with no special provision, i.e., if an overriding function modifies that parameter, ``you get what you get'' (in the virtual function call above, which invokes \tcode{Derived::f}, the postcondition check on \tcode{Base::f} will succeed even though \tcode{Derived::f} does not satisfy the postcondition of \tcode{Base::f}).
\item Allow overriding functions to drop \tcode{const} from a non-reference function parameter, but make it undefined behaviour to actually modify a parameter object in an overriding function if that parameter is a non-reference parameter declared \tcode{const} in an overridden function.
\end{enumerate}

We enumerated the options with a ``V'' prefix (for ``virtual''), to distinguish them from the options from \cite{P3487R0} that have an ``R'' prefix (for ``registers'') and the options from \cite{P3489R0} that have a ``D'' prefix (for ``dependent'').

Below we discuss the tradeoffs of each option.

\subsection*{Option V1}

Option~V1 would prevent the bug in the example above and is the most conservative choice. This choice is consistent with the choice we made for postcondition assertions on coroutines in \cite{P2900R10}: if a function odr-uses a non-reference parameter in its postcondition, and that parameter is declared \tcode{const}, but there might be some other reason why the implementation of the function may modify that parameter anyway, the program is ill-formed. One such reason is that the function is a coroutine and thus the parameter will be modified by the underlying coroutine machinery. Another such reason is that the function is a non-\tcode{final} virtual function and thus an overriding function could modify the parameter.

Just like in the coroutine case, for non-\tcode{final} virtual functions the workaround would be to use a postcondition capture, once this post-MVP feature becomes available (see \cite{P3098R0}):
\begin{codeblock}
struct Base {
  virtual int f(const int i) post (r: r >= i);     // error: cannot odr-use parameter \tcode{i} 
};

struct Base {
  virtual int f(const int i) post [i] (r: r >= i); // OK: explicitly capturing \tcode{i} by copy
}
\end{codeblock}

However, Option~V1 has several downsides.

First, having to capture any parameter in order to odr-use it in the postcondition assertion means that we have to pay the cost of the copy. For coroutines, making that copy is the \emph{only} way to get access to the pre-moved-from value of a parameter in the postcondition; on the other hand, for virtual functions, the copy will be unnecessary in most cases, i.e. we would be paying for the freedom to modify that parameter in an override, but we will most likely not make use of that freedom.

Further, unlike coroutines, any modifications to the parameters must be done explicitly in an overriding function and they will not happen implicitly as part of non-obvious language machinery. This suggests that there is a much lower risk of the user accidentally getting it wrong. For coroutines, there is no way the user could write an implementation that avoids the parameter modification; on the other hand, for virtual functions, there is a very simple way: just do not drop \tcode{const} from the parameter declaration on overriding functions, and do not modify that parameter in the function's implementation. 

Finally, in today's C++ ecosystem, virtual functions are more pervasive than coroutines, and postcondition assertions have more known use cases for virtual functions than for coroutines (the usefulness of postcondition assertions on coroutines is fundamentally limited due to the nature of coroutines in C++). Entirely disallowing the ability to odr-use parameters in the postcondition assertion of a non-\tcode{final} virtual function in the first version of Contracts for C++ that we ship could noticeably hamper the usability of the feature. Shipping postcondition captures as proposed in \cite{P3098R0} in the same version could somewhat mitigate but not fully remove the friction.

Overall, Option~V1 might therefore be a disproportionately harsh measure for a relatively rare problem.

\subsection*{Option V2}

Option~V2 would also prevent the bug in the example above and is a less restrictive choice than Option~V1: it would make only cases ill-formed where the parameter can actually be modified (when the overriding function actually drops the \tcode{const}). This option makes  the behaviour for overriding functions consistent with the behaviour of subsequent declarations of the \emph{same} function: if we redeclare a function, \emph{or} an overload of that function, and drop \tcode{const} on a parameter in that declaration, the program is ill-formed. Option~V2 thus seems more appealing than Option~V1.

However, the tradeoff is that Option~V2 could also lead to remote code breakage, which directly violates Design Principle 15 of \cite{P2900R10}, ``No Client-Side Language Break''. In particular, adding a postcondition to a virtual function that odr-uses a \tcode{const} parameter would remotely break any client code that overrides that function and yet has not added the \tcode{const} to all declarations of the override, including the definition --- including any override that is implemented as a coroutine.

We anticipate that with virtual functions, such remote code breakage would be a frequent problem. Today, hardly anyone adds \tcode{const} to the declaration of a non-reference parameter as it has essentially no meaning; with \cite{P2900R10}, anyone who wishes to odr-use a parameter in a postcondition will \emph{have} to add the \tcode{const}, breaking any overriding functions.

The crucial difference to the coroutine case is that providing a definition for a given function that makes the function a coroutine is local, not remote code, whereas overriding a function can happen in an entirely different component of the program. The possibility of such remote code breakage due to the introduction of Contracts could hamper their adoption and make releasing low-level libraries with newly introduced function-contract assertions significantly more difficult.

\subsection*{Option V3}

Option~V3 would also prevent the bug in the example above, while inflicting a smaller number of required changes to derived types than Option~V2: only the defining declaration of the override will need the \tcode{const} added to the parameter declaration, not \emph{all} declarations of the override.
%In particular, there is a significantly larger chance that derived classes which exist today will compile with no changes.
% TD: I commented this out because I have no idea whether that's true - sounds to me like a hypothesis not backed by any data.
Option~V3 is also likely to inflict a smaller number of remote code breakage, as only the translation unit that contains that defining declaration will be affected. A library that makes use of a derived class with an override affected by the change will itself still compile without changes.

More notably, for users attempting to continue to link in binaries built in days past, as long as the function itself does not modify the function parameters in question (which is the case in the vast majority of cases, even when the parameters are not actually declared \tcode{const}), clients can continue to be rebuilt and linked against those binaries without issue or any need to access and update the source for the derived classes.

Note that Option~V3 does \emph{not} remove the requirement that the parameter be declared \tcode{const} on all declarations of the function that has the postcondition odr-using that parameter. This property of Option~V3 is also arguably its downside: it creates an inconsistency between the rules for the function that has the postcondition and the rules for function overriding it --- the former needs to have \tcode{const} on the parameter declaration on \emph{all} declarations of the function, while the latter need to have \tcode{const} on the parameter declaration only on the \emph{defining} declaration of the function. The other exterme for this alternative to lift the requirement of having \tcode{const} on all declarations, which is the generally non-viable Option~V4 disucssed below.

\subsection*{Option V4}

Option~V4 is a further relaxation of Option~V3, and in fact a relaxation of the current rule in \cite{P2900R10}. With Option~V4, \tcode{const} needs to be present only on the \emph{definition} of the function that defines the postcondition itself, and if that function is a virtual function, also any overriding functions. The requirement that the parameter needs to be declared \tcode{const} on any non-defining declaration of a function would be dropped entirely, including for non-virtual functions. This means the following code, which is ill-formed in \cite{P2900R10} today, would become well-formed:

\begin{codeblock}
int f(int i) post (r: r >=  i);  // OK: \tcode{const} no longer needed here

int f(const int i) {  // \tcode{const} still needed here, ill-formed if missing
  return i;
}
\end{codeblock}
Just like all the preceding options, this option would also prevent the bug in the example above: any program that tries to modify the parameter in the implementation of the function itself or any of its overrides would still be ill-formed. It would inflict the same amount of remote code breakage in overriding clients as Option~V3 (and less than Option~V2). Unlike Option~V3, it would also not introduce any inconsistencies between the requirements on the declarations of the overridden function and its overriding functions, and in fact remove any inconsistencies between virtual and non-virtual functions. The rule for all functions would be the same --- the \tcode{const} just needs to be on the definition.

For functions that \emph{do} have postconditions that odr-use a non-reference parameter, we would need to address the ramifications of allowing an inconsistency between the \tcode{const}-qualifiers on a parameter in the declaration and in the definition.  One might, initially, believe that this inconsistency, which exists today, causes problems that we already deal with acceptably.  For instance, it's possible to use the \tcode{const}-ness of function parametesr in default arguments, \tcode{noexct}-specifiers, and even the definitions of later parameters or trailing return types.   Default parameters can have only one definition in a translation unit and may not odr-use function paramters.  Elsewhere within a declaration the mechanism of choosing elements of a function signature does not matter --- only the result does; i.e., all of the following declarations are for the same function:
\begin{codeblock}
int f(int i, int j);
int f(int i,       std::conditional_t< std::is_const_v<decltype(i)>, long, int> j);
int f(const int i, std::conditional_t<!std::is_const_v<decltype(i)>, long, int> j);
auto f(const int i, int j) -> std::conditional_t<std::is_const_v<decltype(i)>, int, long>;
\end{codeblock}
For function-contract assertions, however, we expect the predicate to be evaluable by both a caller that sees only the declaration of a function and the function itself.  Therefore, we must address what happens when the expressions themselves are parsed with different undestandings of the $cv$-qualifiers of function parameters.  To address this concern, we can consider three possible approaches:
\renewcommand\labelenumi{V4\alph{enumi}.}
\renewcommand\theenumi\labelenumi
\begin{enumerate}
\item Within a function declaration, if a nonreference parameter is odr-used by a postcondition, that parameter shall be impliicitly treated as if it is \tcode{const}.\footnote{This option has been considered and rejected in the past by SG21 when reviewing \cite{P2829R0}. Since this option was first discussed, \cite{P3071R1} has been adopted into the Contracts MVP, often reffered to as \tcode{const}-ification.   With that change, it becomes harder to incorrectly depend on the \tcode{const}-ness of a function parameter, but it will always remain visible with \tcode{decltype}.  Changing the result of \tcode{decltype} is, of course, not an option as that removes any escape hatch from \tcode{const}-ification and breaks any code actually attempting to do computations with the type of a parameter instead of its value.}

\item Apply the one definition rule to the evaluation of any preconditions or postconditions, making it ill-formed, no diagnostic required if the mismatch in \tcode{const} qualifiers on any parameters causes \emph{any} changes in the meanings of the predicates.
\item Allow contract assertions to be evaluated with the parameter declarations from their associated declaration, including $cv$-qualifiers.
\end{enumerate}
       

For a variety of reasons that we will explore, all of these options are highly problematic or unimplementable, therefore we do not consider them viable.

The primary problem arises from the cognitive load on a reader of an API attempting to understand it.  Such readers would be forced to, when reading a postcondition, understand that parameters will be treated as \tcode{const} in the function body without actually seeing that information in the declaration.  Unlike \tcode{const}-ification, this is not about an expression itself not modifying parameters, but the promise that the value a parameter is initialized with when a function is called is still that object's value when the function is returning back to the caller.
% TD: I don't buy this. As a human reader, when reading a declaration, you don't have to care or reason about the definition --- and if you *are* interested in the definition, the const will still be there. So I don't see where the cognitive load comes from. There is indeed cognitive load on treating a parameter as const on *that* declaration even if you didn't write const --- but that cognitive load is already there in P2900 with constification. This doesn't make that worse.
% JMB: No, the cognitive load is that you need to be aware that some parameter initialized when the function is called will have the same value when the function is complete, but nothing on the declaration tells you that this will be the case.  constification doesnt' tell you that will be the case either --- this is about whether the parameter can be modified during the function evaluation, which is unrelated to the evaluation of the contract assertion itself.  

To uncover things that seeing different \tcode{const} qualifiers impacts, we will need to examine the effects when we look past \tcode{const}-ification with \tcode{decltype}, and for that we will be using a simple macro (described more as a useful escape hatch from \tcode{const}-ification in \cite{P3261R1}) to see that effect:
\begin{codeblock}
// A macro to access underlying variable of \tcode{const}-ified names
#define UNCONST(x) const_cast<std::add_lvalue_reference_t<decltype(x)>>(x)
\end{codeblock}

The next problem to consider is how much we want the evaluation of function-contract assertion predicate to evaluate the same functions regardless of where the code for it is generated.  In general in C++, any time the same expression might be interpretted differently we identify such cases as violations of the one defintiion rule and make them ill-formed, no diagnostic required.   We apply the same rules to determine the equivalence of function-contract assertions, and so the following example of two declarations of a function attempting to repeat the preconditions is ill-formed (or IFNDR if the declarations are in separate translation units):
\begin{codeblock}
void f(int i)       pre( std::is_same_v<decltype(i), int> );
void f(const int i) pre( std::is_same_v<decltype(i), const int> );
\end{codeblock}
In general, any precondition or postconditino whose interpretation depends on the \tcode{const} qualifiers of a parameter would be similarly IFNDR if the definition repeated those function-contract assertions.  The various flavors of Option~V4, therefore, come into play when the function-contract assertions are \emph{not} repeated on the definition.

There we must consider what gets invoked when the code for funciton-contract assertions is generated at a call site or within the function body.  Let's look at a function with a precondition that uses a function template passed a parameter through a forwarding reference, and a postcondition that odr-uses that parameter:
\begin{codeblock}
template <typename T>
bool g(T&& t);

int f(int i)  // \tcode{i} does not need to be \tcode{const} here, despite being odr-used in \tcode{post}
  pre(g(UNCONST(i)))   // does this now call \tcode{g<int\&>} instead of \tcode{g<const int\&>}?
  post(r : r > i && g(UNCONST(i)) );
\end{codeblock}
Without the postcondition, \tcode{g<int\&>} is instantiated and called in all circumstances.  With the postcondition our different options will produce different results, none great.


An even more serious problem is that an implementation could now no longer correctly parse the preceding parts of the function declaration until it knows whether a parameter is odr-used in a postcondition. It would have to start parsing the function declaration assuming that the parameter is not \tcode{const}, and then if it sees a postcondition odr-using that parameter it would have to start over. Parsing the function declaration could require instantiating templates, so the compiler might have to tentatively instantiate templates with a non-\tcode{const} type --- which can have side effects such as initialising a static class member --- and then somehow undo that instantiation if it turns out that the parameter actually has a \tcode{const} type because it is odr-used in a postcondition assertion.


With Option~V4a, the call to \tcode{g} would have to be to \tcode{g<const int \&>}, as the use of \tcode{i} in the postcondition would implicitly make \tcode{i} \tcode{const}.   Knowing this, however, does not happen until the postcondition is being parsed, which occurs after the precondition has been parsed and \tcode{g<int \&>} has already been instantiated.   Going back and un-parsing that first attempt to handle the precondition would somehow having kept all side effects of instantiated templates marked in such a way that they could be undone and then re-evaluated finally knowing the expected type of the fuction parameter.   Such rollbacks of template instantiation do not exist in the language today for very good reasons.

Now we should consider the function body of \tcode{f}:
\begin{codeblock}
int f(const int i)  // \tcode{const} required because of postcondition
{
  return i+1;
}
\end{codeblock}
When generating code for this function, and in particular when generating code to check the precondition in this function, we would be doing so for a precondition that was originally parsed where \tcode{i} was of type \tcode{int}, yet the local variable we are applying it to is of type \tcode{const int}.   Either we must force the other declaration to apply to this function parameter, or we must ignore a top-level \tcode{const} qualifier, or we must make this case ill-formed (or IFNDR).

Option~V4b would make this example IFNDR, as the function-contract assertions would have different interpretatinos if attached to different declarations.

Option~V4c would pass this \tcode{const int} object to \tcode{g<int \&>}.  This falls apart if \tcode{g<int \&>} actually does modify its parameter, as then the modification would be happening to a variable whose declaration has a top-level \tcode{const}-qualifier, which is udnefined behavior.  Worse, even though such parameters are unlikely to be placed in read-only storage, the soundness of the postcondition relies on that value not changing.

Option~V4a has even more direct problems if a \tcode{const_cast} is used without the use of \tcode{decltype}:
\begin{codeblock}
int f(int i) 
  pre(++const_cast<int&>(i))  // modifying \tcode{i} is now UB here...
  post( r : r > i);  // ...because \tcode{post} is odr-using \tcode{i} here!
\end{codeblock}

Option~V4a is further flawed due to the fact that even the fact that a postcondition odr-uses a parameter is itself something that can be dependent on the \code{const}-ness of the parameter, leading to paradoxical situations:
\begin{codeblock}
void f(int i)
  post([&]() {
    if constexpr (std::is_const_v<decltype(i)>) {
      return true; 
    } else {
      return i != 0;
    }
  }());
\end{codeblock}

It is our current view that the above problems render all of these approaches to dropping the \tcode{const} requirement on declarations that have a postcondition not practicably implementable and not clearly specifiable. 

\subsection*{Option V5}

Option~V5 is the status quo\footnote{Since the current specification in \cite{P2900R10} does not contain any special rules for parameter declarations on overriding functions, dropping \tcode{const} in such declarations is currently allowed and ``you get what you get''.} in \cite{P2900R10}. It avoids the usage limitations imposed by Option~V1 and the remote code breakage imposed by Option~V2 or Option~V3 and does not suffer from the implementation issues of Option~V4. However, this option too has tradeoffs.

One downside of Option~V5 is that, unlike Options V1 --- V4, it would not actually prevent the bug, as it would allow a program to modify the parameter value in the implementation of an override. As a partial remedy, an implementation could easily issue a \emph{warning} if an overriding function drops the \tcode{const} on a parameter odr-used in the postcondition assertion of an overridden function. The crucial difference to the coroutine case is that the implementation \emph{knows} that the function is virtual at the point of declaration. While we cannot normatively mandate such a warning, we can add a non-normative recommended practice note to the wording that such a warning be issued. 

Another downside of Option~V5 compared to Options V1 -- V4 is that it would mean abandoning the static guarantee of the current \cite{P2900R10} design that, if a parameter is odr-used in a postcondition assertion, that parameter object \emph{will not be involved in any non-\tcode{const} operations} and, for many types, will thus \emph{not be modified} between the call to a function and the evaluation of its postcondition assertions.

It follows that neither humans nor static analysis tools would be able to reason about the value of a function parameter in a postcondition, at least not without taking into account which function will be selected by virtual dispatch and what the body of that function does (both of which are typically unknowable at the call site). This greatly reduces the amount of useful information that a static analysis tool could extract from a postcondition assertion at the call site of a virtual function.

\subsection*{Option V6}

Option~V6 is the option chosen in the past by C++2a Contracts \cite{P0542R5}. However, we consider it completely unviable. Not only would this option fail to prevent the bug in the example above, but it would make the situation even worse: not only would there be a broken postcondition, but the program would also have undefined behaviour. This approach is therefore actively user-hostile and violates Design Principle 13 of \cite{P2900R10}, ``Explicitly Define All New Behaviour''.

%JMB: I think this would benefit from a subsection or lead-in paragraph.
%     I also think that visually it makes V4 and V6 look appealing unless you think about it --- maybe if we grayed out those columns since we are not proposing those options?   
%%%%%%%%%%%%%%%%%%%%%%%%%%%%%%%%%%%%%%%%%%%%%
\newcommand{\yes}{\includegraphics[width=4mm]{images/yes.png}}
\newcommand{\no}{\includegraphics[width=4mm]{images/no.png}}
\newcommand{\maybe}{\includegraphics[width=4mm]{images/maybe.png}}
%\vspace{4mm}
\begin{table}[t]
\begin{tabular}{|p{8cm}|p{0.9cm}|p{0.9cm}|p{0.9cm}|p{0.9cm}|p{0.9cm}|p{0.9cm}|}
\hline 
& V1 & V2 & V3 & V4 & V5 & V6 \\
\hline
Allows non-reference parameters to be odr-used in \tcode{post} on virtual functions & \no & \yes  & \yes  & \yes & \yes & \yes\\ \hline
Prevents modification of non-reference parameters odr-used in \tcode{post} and associated bugs & \yes & \yes  & \yes  & \yes & \no & \no\\ \hline
Adding \tcode{post} to a virtual function does not  break derived classes that override it& \yes & \no  & \no  & \no & \yes & \yes \\ \hline
Adding \tcode{post} to a virtual function does not break users of a derived class that overrides it & \yes & \no  & \yes  & \yes & \yes & \yes\\ \hline
Does not introduce new inconsistencies between overriding and non-overriding functions & \yes & \yes  & \no  & \yes & \yes & \yes\\ \hline
No significant concerns regarding specifiability and implementability* & \yes & \yes  & \yes  & \no & \yes & \yes\\ \hline
Does not introduce new sources of undefined behaviour* & \yes & \yes  & \yes  & \yes & \yes & \no\\ \hline
\end{tabular}
\vspace{2mm}
\caption{Main tradeoffs of proposed options V1 --- V6. Options that do not satisfy a requirement marked with an asterisk* are considered unviable by the authors of this paper.}
\label{tradeoffs}
\end{table}

\section{Proposal}

We believe that Options V1 --- V3 as well as V5 are worth considering, whereas Options V4 and V6 are completely unviable for the reasons listed above. We therefore propose Options V1 --- V3 as well as V5 to determine which option has more consensus in SG21. 

Note that choosing Option~V1 would leave the door open to adopting V2, V3, or V5 without breaking changes at some point in the future, while Option~V2 could only be evolved towards Option~V3 or Option~V5, Option~V3 could only be evolved towards Option~V5, and Option~V5 could not be evolved towards either of the other options without breaking changes.

Table~\ref{tradeoffs} summarises the main tradeoffs of all options V1 --- V6 discussed in the paper.

\section{Wording}

The proposed wording changes are relative to the wording proposed in \cite{P2900R10}. Note that we do not explicitly call out the case of an overriding function being implemented as a coroutine, however as per the wording proposed in \cite{P2900R10}, a coroutine behaves as if the top-level $cv$-qualifiers in all parameter-declarations in the declarator of its defining declaration were removed.

\subsection*{Option~V1}

Modify [dcl.contract.func] as follows:

\begin{adjustwidth}{0.5cm}{0.5cm}
If the predicate of a postcondition assertion of a function odr-uses ([basic.def.odr]) a
non-reference parameter of that function, all declarations of that parameter shall have a \tcode{const} qualifier and shall not have array or function type\added{; if the function is virtual
it shall be marked with the \grammarterm{virt-specifier} \tcode{final} (see [class.virtual]) or it shall be a a member function of a class with the
\grammarterm{class-virt-specifier} \tcode{final} (see [class.pre])}.
\begin{note}
This requirement applies even to declarations
that do not specify the \grammarterm{postcondition-specifier}. Arrays and functions are still usable when declared with the equivalent pointer types ([dcl.fct]).
\end{note}
\begin{example}
\tcode{[...]}
\end{example}
\end{adjustwidth}

\subsection*{Option~V2}

Modify [dcl.contract.func] as follows:

\begin{adjustwidth}{0.5cm}{0.5cm}
If the predicate of a postcondition assertion of a function \added{$f$ }odr-uses ([basic.def.odr]) a
non-reference parameter of \removed{that function}\added{$f$}, all declarations of that parameter \added{and the corresponding parameter on any functions that override $f$ }shall have a \tcode{const} qualifier and and shall not have array or function type.
\begin{note}
This requirement applies even to declarations
that do not specify the \grammarterm{postcondition-specifier}. Arrays and functions are still usable when declared with the equivalent pointer types ([dcl.fct]).
\end{note}
\begin{example}
\tcode{[...]}
\end{example}
\end{adjustwidth}

\subsection*{Option~V3}

Modify [dcl.contract.func] as follows:

\begin{adjustwidth}{0.5cm}{0.5cm}
If the predicate of a postcondition assertion of a function \added{$f$ }odr-uses ([basic.def.odr]) a
non-reference parameter of \removed{that function}\added{$f$}, all declarations of
that parameter shall have a \tcode{const} qualifier and shall not have array or function type\added{; the corresponding parameter declaration in the definition of any function $g$ that overrides $f$ shall hae a \tcode{const}
qualifier and shall not have array or function type}.
\begin{note}
This requirement applies even to declarations
that do not specify the \grammarterm{postcondition-specifier}. Arrays and functions are still usable when declared with the equivalent pointer types ([dcl.fct]).  For overrides that are coroutines, this requirement applies to the 
\end{note}
\begin{example}
\tcode{[...]}
\end{example}
\end{adjustwidth}

\subsection*{Option~V5}

Modify [dcl.contract.func] as follows:

\begin{adjustwidth}{0.5cm}{0.5cm}
If the predicate of a postcondition assertion of a function odr-uses ([basic.def.odr]) a
non-reference parameter of that function, all declarations of that parameter shall have a \tcode{cons} qualifier and
shall not have array or function type.
\begin{note}
This requirement applies even to declarations
that do not specify the \grammarterm{postcondition-specifier}. Arrays and functions are still usable when declared with the equivalent pointer types ([dcl.fct]).
\end{note}
\begin{example}
\tcode{[...]}
\end{example}

\added{\emph{Recommended practice:} Implementations should issue a diagnostic when an overriding function omits \tcode{const} from any declaration of a non-reference parameter whose corresponding parameter in an overridden function is odr-used in a postcondition assertion of that overridden function.}
\end{adjustwidth}

%%%%%%%%%%%%%%%%%%%%%%%%%%%%%%%%%%%%%%%%%%%%%

\section*{Acknowledgements}

Thanks to John Lakos and Oliver Rosten for their review of the paper.

%%%%%%%%%%%%%%%%%%%%%%%%%%%%%%%%%%%%%%%%%%%%%

\section*{Revision history}

\begin{itemize}
\item \textbf{R0}, 2024-10-31: Original version as presented to SG21 on 2024-10-31
\item \textbf{R1}, 2024-11-07: Incorporated SG21 feedback; explained relationship of this paper to companion papers \cite{P3487R0} and \cite{P3489R0}; prefixed proposal numbers with ``V'' to be distinct from companion papers; various minor fixes
\item \textbf{R2}, 2024-11-09: Introduced Option~V3 to only require \tcode{const} on definitions of overrides and Option~V4 to only require \tcode{const} on definitions of any functions; added deeper explanation of remote breakage scenarios; added table summarising tradeoffs
\end{itemize}

%\pagebreak

%%%%%%%%%%%%%%%%%%%%%%%%%%%%%%%%%%%%%%%%%%%%%

% Remove ToC entry for bibliography
\renewcommand{\addcontentsline}[3]{}% Make \addcontentsline a no-op to disable auto ToC entry

%\renewcommand{\bibname}{References}  % custom name for bibliography
\bibliographystyle{abstract}
\bibliography{ref}

%%%%%%%%%%%%%%%%%%%%%%%%%%%%%%%%%%%%%%%%%%%%%


\end{document}

